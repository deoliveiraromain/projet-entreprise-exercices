\documentclass[12pt]{article}
\usepackage[a4paper]{geometry}
\usepackage[utf8]{inputenc}
\usepackage[english,frenchb]{babel}
\usepackage[T1]{fontenc}
\geometry{hscale=0.8,vscale=0.8,centering} %%% change les marges
%\usepackage{wrapfig}  %%% pour utiliser wrapfigure
\usepackage{graphicx} %%% package pour les figures
\usepackage{multirow} %%% ajout de multi ligne dans les tableaux
\usepackage{fancybox}
\usepackage{ccaption}
\usepackage{color}
\usepackage{lmodern}
\usepackage{fancyvrb}
\usepackage{epigraph} %%%% ajout de citation
\usepackage[babel=true]{csquotes} % util d'enquote pour les guillemets (indépendant de la langue)
\usepackage{fancyhdr} %%% changement du style de la page
\usepackage{fourier} %%% changement de la police
\usepackage[nottoc, notlof, notlot]{tocbibind} %%% bibiliographie table des matières
\usepackage{enumerate} % add enumeration style

%Options: Sonny, Lenny, Glenn, Conny, Rejne, Bjarne, Bjornstrup
\usepackage[Sonny]{fncychap} %%% style des noms des champitres
\usepackage{listings} %%%% mettre du code source dans le rapport
\usepackage{caption}
\usepackage{xcolor} 
\ChNameVar{\vspace*{-0.9in}\Large\sf} %%% espacement du chapitre

%%%%%% Ajoute la mention confidentiel sur la première page %%%%%%%%%
%\usepackage[firstpage]{draftwatermark} 
%\SetWatermarkText{\textsc{Confidentiel}} %%%% mention que l'on veut mettre
%\SetWatermarkScale{0.5}                  %%%% taille
%\SetWatermarkColor[rgb]{0.7,0,0}         %%%% couleur
%\SetWatermarkAngle{45}                   %%%% angle

%%%%%%%%%%%%%%%%% Profondeur du compteur %%%%%%%%%%%%%%%%
\setcounter{secnumdepth}{5}
\setcounter{tocdepth}{5}

%%%%%%%%%%%%%%%%%%%% Lien vers la bibiograhie %%%%%%%%%%%%%%%%%%%%%%
%%%%%% A placer à la fin des packages pour créer tous les liens %%%%
\usepackage[colorlinks=true]{hyperref}
\hypersetup{urlcolor=black,linkcolor=black,citecolor=black,colorlinks=true}
%%%%%%%%% Ajoute le glossaire dans la bibliographie %%%%%%%%%%%%%%%%
\usepackage[toc]{glossaries} 

\pagestyle{fancy}
\fancyhead[RE]{\textit{\nouppercase{\leftmark}}}
\fancyhead[LO]{\textit{\nouppercase{\rightmark}}}
 
\fancypagestyle{plain}{ %
\renewcommand{\headrulewidth}{0pt} 
\renewcommand{\footrulewidth}{0pt}}

%%%%%%%%%%%% Enumération %%%%%%%%%%%%%%%%
%%%%%%%%%%%% Nombre puis romain %%%%%%%%%
\renewcommand{\theenumii}{\roman{enumii}}
\renewcommand{\labelenumii}{\theenumii)}

\makeatletter
\def\clap#1{\hbox to 0pt{\hss #1\hss}}%
\def\ligne#1{%
\hbox to \hsize{%
\vbox{\centering #1}}}%
\def\haut#1#2#3{%
\hbox to \hsize{%
\rlap{\vtop{\raggedright #1}}%
\hss
\clap{\vtop{\centering #2}}%
\hss
\llap{\vtop{\raggedleft #3}}}}%
\def\bas#1#2#3{%
\hbox to \hsize{%
\rlap{\vbox{\raggedright #1}}%
\hss
\clap{\vbox{\centering #2}}%
\hss
\llap{\vbox{\raggedleft #3}}}}%
\def\maketitle{%
\thispagestyle{empty}\vbox to \vsize{%
\haut{}{\@blurb}{}
\vfill
\vfill
\vspace{1cm}
\begin{flushleft}
\usefont{OT1}{ptm}{m}{n}
\huge \@title
\end{flushleft}
\par
\hrule height 4pt
\par
\begin{flushright}
\usefont{OT1}{phv}{m}{n}
\Large \@author
\par
\end{flushright}
\begin{flushright}
\vfill
%\Large{\textbf{\@societe}}
%\par
%\Large{\@responsable}
%\par
%\Large{\@adresseA}
%\par
%\Large{\@adresseB}
\end{flushright}
\vspace{1cm}
\vfill
\vfill
\bas{}{\@location, le \@date}{}
}%
\cleardoublepage
}
\def\date#1{\def\@date{#1}}
\def\author#1{\def\@author{#1}}
\def\title#1{\def\@title{#1}}
\def\responsable#1{\def\@responsable{#1}}
\def\location#1{\def\@location{#1}}
\def\societe#1{\def\@societe{#1}}
\def\adresseA#1{\def\@adresseA{#1}}
\def\adresseB#1{\def\@adresseB{#1}}
\def\blurb#1{\def\@blurb{#1}}
\date{\today}
\author{Bastien Charès\\Romain De Oliveira\\Maxime Gautré}
\title{Documentation technique}
%\responsable{Responsable~: Julien Revault d'Allonnes}
%\societe{2MoRO Solutions}
%\adresseA{Pavillon d'Izarbel - Technopole Izarbel}
%\adresseB{64210 Bidart (France)}
\location{Pau}
\blurb{%
E.I.S.T.I \\
Ecole Internationale des Sciences \\du Traitement de l'Information\\
\textbf{Projet d'entreprise}\\[1em]}

\newskip\@bigflushglue \@bigflushglue = -100pt plus 1fil

\def\bigcenter{\trivlist \bigcentering\item\relax}
\def\bigcentering{\let\\\@centercr\rightskip\@bigflushglue%
\leftskip\@bigflushglue
\parindent\z@\parfillskip\z@skip}
\def\endbigcenter{\endtrivlist}

%%%%%%%%%%%%%%%%%%%%%%%%% Glossaire %%%%%%%%%%%%%%%%%
\renewcommand*{\glossaryname}{Glossaire}
\renewcommand*{\glspostdescription}{}
\deftranslation{Glossary}{Glossaire}
\makeglossaries

%%%%%%%%%%%%%%%%%%%%%%%%% Code source listings %%%%%%%%%%%%%%%
\definecolor{dkgreen}{rgb}{0,0.4,0}
\definecolor{gray}{rgb}{0.5,0.5,0.5}
\definecolor{mauve}{rgb}{0.48,0,0.72}
\definecolor{myblue}{rgb}{0.1,0.1,0.8}
\definecolor{rltgreen}{rgb}{0.13,0.54,0.13}
\definecolor{rltgris}{rgb}{0.5,0.5,0.4}


% "define" Scala
\lstdefinelanguage{scala}{
  morekeywords={abstract,case,catch,class,def,%
    do,else,extends,false,final,finally,%
    for,if,implicit,import,match,mixin,%
    new,null,object,override,package,%
    private,protected,requires,return,sealed,%
    super,this,throw,trait,true,try,%
    type,val,var,while,with,yield},
  otherkeywords={=>,<-,<\%,<:,>:,\#,@},
  sensitive=true,
  morecomment=[l]{//},
  morecomment=[n]{/*}{*/},
  morestring=[b]",
  morestring=[b]',
  morestring=[b]"""
}

% Default settings for code listings
\lstdefinestyle{customScala}
  {frame=tb,
  language=scala,
  aboveskip=3mm,
  belowskip=3mm,
  showstringspaces=false,
  columns=flexible,
  basicstyle={\small\ttfamily},
  numbers=left,
  numberstyle=\small\color{gray},
  keywordstyle=\color{myblue},
  commentstyle=\color{dkgreen},
  stringstyle=\color{mauve},
  breaklines=true,
  breakatwhitespace=true,
  tabsize=3,
  literate={á}{{\'a}}1 {ã}{{\~a}}1 {é}{{\'e}}1 {è}{{\`e}}1 {î}{{\^{i}}}1 {â}{{\^{a}}}1 
}

% "define" Javascript 
\lstdefinelanguage{javaScript}{
        keywords={typeof, new, true, false, catch, function, return, null, catch, switch, var, if, in, while, do, else, case, break},
        keywordstyle=\color{blue}\bfseries,
        ndkeywords={class, export, boolean, throw, implements, import, this},
        ndkeywordstyle=\color{darkgray}\bfseries,
        identifierstyle=\color{black},
        sensitive=false,
        comment=[l]{//},
        morecomment=[s]{/*}{*/},
        commentstyle=\color{purple}\ttfamily,
        stringstyle=\color{red}\ttfamily,
        morestring=[b]',
        morestring=[b]"}
        
\lstdefinestyle{customJavascript}
  {frame=tb,
  language=javaScript,
  aboveskip=3mm,
  belowskip=3mm,
  showstringspaces=false,
  columns=flexible,
  basicstyle={\small\ttfamily},
  numbers=left,
  numberstyle=\small\color{gray},
  keywordstyle=\color{myblue},
  commentstyle=\color{dkgreen},
  stringstyle=\color{mauve},
  breaklines=true,
  breakatwhitespace=true,
  tabsize=3,
  literate={à}{{\`a}}1 {ã}{{\~a}}1 {é}{{\'e}}1 {è}{{\`e}}1 {\$}{{\textcolor{red}{\$}}}1
}
% customXml
\lstdefinestyle{customXml}
  {frame=tb,
  language=XML,
  aboveskip=3mm,
  belowskip=3mm,
  showstringspaces=false,
  columns=flexible,
  basicstyle={\small\ttfamily},
  numbers=left,
  numberstyle=\small\color{gray},
  keywordstyle=\color{myblue},
  commentstyle=\color{dkgreen},
  stringstyle=\color{mauve},
  breaklines=true,
  breakatwhitespace=true,
  tabsize=3,
  literate={á}{{\'a}}1 {ã}{{\~a}}1 {é}{{\'e}}1 {è}{{\`e}}1
}

% "define" json
\colorlet{punct}{red!60!black}
\definecolor{delim}{RGB}{20,105,176}
\colorlet{numb}{magenta!60!black}
\lstdefinelanguage{json}{
    aboveskip=3mm,
    belowskip=3mm,
    showstringspaces=false,
    columns=flexible,
    basicstyle={\small\ttfamily},
    numbers=left,
    numberstyle=\small\color{gray},
    showstringspaces=false,
    breaklines=true,
    breakatwhitespace=true,
    tabsize=3,
    frame=tb,
    literate=
     *{0}{{{\color{numb}0}}}{1}
      {1}{{{\color{numb}1}}}{1}
      {2}{{{\color{numb}2}}}{1}
      {3}{{{\color{numb}3}}}{1}
      {4}{{{\color{numb}4}}}{1}
      {5}{{{\color{numb}5}}}{1}
      {6}{{{\color{numb}6}}}{1}
      {7}{{{\color{numb}7}}}{1}
      {8}{{{\color{numb}8}}}{1}
      {9}{{{\color{numb}9}}}{1}
      {:}{{{\color{punct}{:}}}}{1}
      {,}{{{\color{punct}{,}}}}{1}
      {\{}{{{\color{delim}{\{}}}}{1}
      {\}}{{{\color{delim}{\}}}}}{1}
      {[}{{{\color{delim}{[}}}}{1}
      {]}{{{\color{delim}{]}}}}{1},
}

%%%%%%%%%%%%%%%%% Remerciements même forme qu'un abstract %%%%%%%%%%%%
\newenvironment*{remerciements}{%
\renewcommand*{\abstractname}{Remerciements}
\begin{abstract}
}{\end{abstract}}

%%%%%%%%%%%%%%%%% liste des figures en tant que section %%%%%%%%%%%%
\renewcommand\listoffigures{%
    \section{\listfigurename}% Used to be \section*{\listfigurename}
      \@mkboth{\MakeUppercase\listfigurename}%
              {\MakeUppercase\listfigurename}%
    \@starttoc{lof}%
    }
